\documentclass[11pt,fleqn]{article}

\usepackage[utf8]{inputenc}
\usepackage{latexsym}
\usepackage{pslatex}
\usepackage{xspace}
\usepackage{url}

\title{PlusCal / \tlaplus: An Annotated Cheat Sheet}
\author{Stephan Merz\\ \url{stephan.merz@loria.fr}}
\date{}

\newcommand{\tlaplus}{TLA\textsuperscript{+}\xspace}
\newcommand{\kw}[1]{\textnormal{\textsc{#1}}}
\newcommand{\deq}{\,\mathrel{\smash{\stackrel{\scriptscriptstyle\Delta}{=}}}\,}
\newcommand{\seq}[1]{\ensuremath{\langle #1 \rangle}}
\newcommand{\str}[1]{\textnormal{"#1"}}
\newcommand{\TRUE}{\kw{true}}
\newcommand{\FALSE}{\kw{false}}
\newcommand{\CHOOSE}{\kw{choose}}
\newcommand{\UNION}{\kw{union}}
\newcommand{\SUBSET}{\kw{subset}}
\newcommand{\DOMAIN}{\kw{domain}}
\newcommand{\EXCEPT}{\kw{except}}
\newcommand{\IF}{\kw{if}}
\newcommand{\THEN}{\kw{then}}
\newcommand{\ELSE}{\kw{else}}
\newcommand{\LET}{\kw{let}}
\newcommand{\IN}{\kw{in}}
\newcommand{\CASE}{\kw{case}}
\newcommand{\OTHER}{\kw{other}}
\newcommand{\RECURSIVE}{\kw{recursive}}
\newcommand{\EXTEND}{\kw{extend}}
\newcommand{\INSTANCE}{\kw{instance}}
\newcommand{\CONSTANTS}{\kw{constants}}
\newcommand{\VARIABLES}{\kw{variables}}
\newcommand{\ASSUME}{\kw{assume}}


\begin{document}
\maketitle

\begin{abstract}
  This document is intended to summarize the main constructs that a user of
  PlusCal and \tlaplus who is using the Toolbox and TLC is likely to encounter.
  It does not replace the available introductory material about \tlaplus and is
  also not meant as a reference manual. Feedback to the author is welcome.
\end{abstract}

\section{PlusCal}

PlusCal is an algorithmic language that has the ``look and feel'' of imperative
pseudo-code for describing concurrent algorithms. It has a formal semantics,
through translation to \tlaplus, and algorithms can be verified using the
\tlaplus tools. We describe here the ``C syntax'' of PlusCal, but there is also
a ``P syntax'' closer to the Pascal programming language.

A PlusCal algorithm appears inside a comment within a \tlaplus module. Its
top-level syntax is as follows.

\begin{verbatim}
--algorithm name {
  (* declaration of global variables *)
  (* operator definitions *)
  (* macro definitions *)
  (* procedures *)
  (* algorithm body or process declarations *)
}
\end{verbatim}

Variable declarations, operator and macro definitions, as well as procedures are
optional. There must either be an algorithm body (for a sequential algorithm) or
process declarations (for a concurrent algorithm).

The PlusCal translator embeds the \tlaplus specification corresponding to the
PlusCal algorithm in the module within which the algorithm appears, immediately
following the algorithm. The translation is delimited by the lines
%
\begin{verbatim}
\* BEGIN TRANSLATION
...
\* END TRANSLATION
\end{verbatim}
%
Users should not touch this area, as it will be overwritten whenever the PlusCal
translator is invoked. However, \tlaplus definitions may appear either before
the PlusCal algorithm or after the generated \tlaplus specification. In
particular, operators defined before the PlusCal algorithm may be used in the
algorithm. Correctness properties are defined below the generated specification
because they refer to the variables declared by the translator.


\paragraph{Variable declarations.}

Variables are declared as follows, they may (but need not) be initialized:

\verb|variables x, y=0, z \in {1,2,3};|

Note that there may be at most one ``variables'' clause, but that it may declare
any number of variables. This example declares three variables $x$, $y$, and
$z$. The variable $x$ is not initialized, $y$ is initialized to $0$, and $z$ may
take either $1$, $2$ or $3$ as initial values. During model checking, TLC will
assign $x$ a default value that is different from all ordinary values and will
explore all three initial values for $z$.

\paragraph{Operator definitions.}

As in \tlaplus (see below), operators represent utility functions for describing
the algorithm. The syntax is the same as for \tlaplus definitions, explained
below. For example,

\begin{verbatim}
define {
  Cons(x,s) == << x >> \o s
  Front(s)  == [i \in 1 .. Len(s)-1 |-> s[i]]
}
\end{verbatim}

declares an operator $\mathit{Cons}$ that prepends an element $x$ to a sequence
$s$ and an operator $\mathit{Front}$ that returns the subsequence of a non-empty
sequence without the last element. Again, there can be a single ``define''
clause, but it may contain any number of operator definitions, without any
separator symbol.

\paragraph{Macros.}

A macro represents several PlusCal statements that can be inserted into an
algorithm, and it may have parameters. In contrast to a defined operator, a
macro need not be purely functional but may have side effects. In contrast to a
procedure, it cannot be recursive and may not contain labels or complex
statements such as loops or procedure calls or return statements. Here are two
examples:

\begin{verbatim}
macro send(to, msg) {
  network[to] := Append(@, msg)
}
macro rcv(p, var) {
  await Len(network[p]) > 0;
  var := Head(network[p]);
  network[p] := Tail(@)
}
\end{verbatim}

These macros represent send and receive operations in a network represented by
FIFO queues indexed by processes. The first macro appends a message to the queue
of the destination process. The second macro blocks until the queue of process
$p$ contains at least a message, then assigns the first message in the queue to
variable $\mathit{var}$ and removes it from that queue. When using a macro,
simply insert an invocation as a statement in the code, such as
\verb|send(p, x+1)| for sending the value $x+1$ to process $p$.

\paragraph{Procedures.}

A procedure declaration is similar to that of a macro, but it may also contain
the declaration of local variables.\footnote{These variables may be initialized
  using the form $x = v$, but not $x \in S$.} The procedure body may contain
arbitrary statements, including procedure calls (even recursive calls).

\begin{verbatim}
procedure p(x,y)
    variable z=x+1  \* procedure-local variable
{
    call q(z,y);
l:  y := y+1;
    return
}
\end{verbatim}

Procedure parameters are treated as variables and may be assigned to. Any
control flow in a procedure should reach a return statement, any result computed
by the procedure should be stored in a variable---possibly a parameter.
Procedure calls are preceded by the keyword call and must be followed by a
label.

Since the PlusCal translator introduces a stack for handling procedures, a
PlusCal algorithm using procedures must appear in a module \EXTEND{}ing the
standard module \verb|Sequences|.

\paragraph{Processes.}

A PlusCal algorithm may declare one or more process templates, each of which can
have several instances. A process template looks as follows:

\begin{verbatim}
process (name [= | \in] e)
   variables ...   \* process-local variables
{
   (* process body *)
}
\end{verbatim}

The form ``$name = e$'' will give rise to one instance of the process whose
process identity is (the value of expression) $e$, whereas ``$name \in e$'' will
create one process instance per element of the set $e$. When creating several
process instances from different templates, it is important to ensure that all
process identities are comparable and different: for example, use only integers,
only strings or only model values. Within the process body, the process identity
is referred to as \verb|self| (and not as \verb|name| as the syntax might
suggest). Processes conceptually execute (asynchronously) in parallel, from the
start of execution of the algorithms: PlusCal does not support dynamically
spawning processes during execution.


\paragraph{PlusCal statements.}

The statements of PlusCal are those expected of a simple imperative language,
with the addition of primitives for synchronization and for non-deterministic
choice that are useful for specifications.

Assignments are written \verb|x := e|, but PlusCal also supports assignments to
components of an array \verb|x[i,j] := e| whose translation involves \EXCEPT{}
forms in \tlaplus. PlusCal also has a statement \verb|print e| for printing the
value of an expression.\footnote{Use of this statement requires \EXTEND{}ing the
  standard module TLC.} However, due to the breadth-first state enumeration
strategy used by default in TLC, it may not always be obvious to understand
the relationship between outputs to the console and actual execution flow.

The conditional statement has the usual form
%
\begin{verbatim}
   if (exp) ... else ...
\end{verbatim}
%
where the ``else'' branch is optional; compound statements are enclosed in
braces. Similarly, while loops are written
%
\begin{verbatim}
   while (exp) ...
\end{verbatim}

Other statements for transferring the flow of control are procedure call and
return (see above), as well as \verb|goto l| for jumping to label $l$---see
below for a more detailed explanation of labels in PlusCal.

Assertions can be embedded in PlusCal in the form \verb|assert e|. This
statement has no effect if the predicate $e$ is true and otherwise aborts
(producing a counter-example trace in TLC). There is also \verb|skip|, which
does nothing and is equivalent to \verb|assert TRUE|.

The statement \verb|await e| (which can alternatively be written \verb|when e|)
blocks execution until the condition is true. This is useful for synchronizing
parallel processes, for example for blocking message reception until a message
has actually been received.

Finally, PlusCal offers two statements for modeling non-determinism. The first
form,
%
\begin{verbatim}
  either \* first choice
  or     \* second choice
  or     \* ...
\end{verbatim}
%
is useful for expressing the choice between a fixed number of alternatives. A
useful idiom for restricting this choice is to add \verb|await| statements to
branches: only those choices whose condition evaluate to \TRUE{} can indeed be
executed. (This is PlusCal's version of Dijkstra's guarded commands.)

Second, the form
%
\begin{verbatim}
   with (x \in S) ...
\end{verbatim}
%
allows for choices based on the elements of set $S$: the variable $x$ acts as a
local variable whose scope is restricted to the body of the \verb|with|
statement. For example, in a model where communication is not necessarily FIFO,
$S$ could be the set of messages that a process has received, and a \verb|with|
statement could be used to handle any one of these messages.

\paragraph{Semicolons.}

Unlike in C and similar languages, PlusCal uses semicolons to separate (and not
end) statements. In particular, this means that no semicolon is required before
the closing brace of a compound statement (although it is not considered as a
syntax error). However, a semicolon is required \emph{after} the closing brace
if it is followed by another statement.

\paragraph{Labels.}

PlusCal statements can be labeled, and the primary purpose of labels is to
indicate the ``grain of atomicity'' in the execution of parallel processes. The
idea is that a group of statements that appears between two labels is executed
without interruption by any other process. However, the PlusCal translator
imposes certain labeling rules. The most important ones are:
\begin{itemize}
\item The first statement in the body of a procedure, a process or of the
  algorithm body must be labeled.
\item A \verb|while| statement must be labeled.
\item Any statement following a \verb|call|, a \verb|return| or an \verb|if| or
  \verb|either| statement that contains a labeled statement must be labeled.
\item A macro body and the body of a \verb|with| statement may not contain
  labels.
\item Any two assignments to the same variable (including to two components of
  the same array) must be separated by a label.
\end{itemize}
The full set of labeling rules is given in the PlusCal manual. The translator
prints information about missing labels, it may also be run with the
\verb|-label| option to automatically add missing labels.

\paragraph{Specifying Fairness.}

Fairness conditions require that statements that are ``often enough'' enabled
must eventually be executed; they are necessary in order to ensure that an
algorithm satisfies any liveness property. In PlusCal, fairness conditions can
be attached to the algorithm, process templates or labels. By writing
\verb|fair algorithm|, one requires that some statement (for some process) must
eventually be executed if some statement is enabled. This is a weak condition:
in particular, some process may never execute although it is always enabled, if
other processes are also always enabled. Writing \verb|fair process| ensures
that the process (more precisely, each instance of the process template) will
eventually execute if it remains enabled. The even stronger condition
\verb|fair+ process| requires strong fairness for the process, i.e.\ it will
eventually execute if it is infinitely often enabled---even if it is also
infinitely often disabled. For example, a process that requires a semaphore that
is infinitely often free will eventually acquire it under strong fairness, even
in the presence of competing processes.

The overall fairness requirement attached to a process can be modulated for
individual actions (corresponding to a group of statements introduced by a
label). Writing \verb|l:-| instead of \verb|l:| suppresses the fairness
assumption for the group of statements introduced by label \verb|l|. For
example, in a mutual-exclusion algorithm, one may not want to assume any
fairness condition for the non-critical section. On the contrary, writing
\verb|l:+| assumes strong fairness for that group of statements. (This has no
effect for a strongly fair process.) For example, one may in this way require
strong fairness only for acquiring a semaphore within a weakly fair process.

Even finer-grained fairness conditions can be expressed in \tlaplus, details can
be found in the literature on \tlaplus.



\section{\tlaplus}

PlusCal algorithms are translated into the \tlaplus specification language, and
the \tlaplus tools (in particular the \tlaplus Toolbox and the model checker
TLC) are used for verifying properties of algorithms. \tlaplus is based on
ordinary mathematical set theory, and it is untyped. In order to effectively use
PlusCal, you need to know at least some \tlaplus syntax:
\begin{itemize}
\item all expressions that appear in PlusCal algorithms are written in \tlaplus,
  and
\item properties to be verified (beyond assertions inserted into PlusCal code)
  are also written in \tlaplus.
\end{itemize}
\tlaplus has ASCII syntax but can be pretty-printed from the Toolbox, using
standard mathematical notation. We show the pretty-printed versions as well as
the ASCII source below.


\subsection{Overall structure}

\paragraph{Modules and declarations.}

\tlaplus specifications are structured in \emph{modules}. A \tlaplus module
begins with a header line

\verb|---------- MODULE Name ----------|

\noindent%
(at least four `-' signs) and ends with a bottom line

\verb|=================================|

\noindent%
(at least four `=' signs). A module may import the contents of other modules by
\EXTEND{}ing them, which is basically equivalent to copying the content of the
extended module into the extending module. It may also create \INSTANCE{}s of
other modules whose (constant or variable) parameters are instantiated by
expressions of the instantiating module. This is useful for example when
showing that a lower-level specification refines (or implements) a higher-level
one.

Modules usually declare constant or variable parameters,\footnote{The values of
  constant parameters remain fixed throughout any execution of the algorithm.
  \tlaplus variables are analogous to program variables and assume different
  values at different states of the execution.} for example
%
\[\begin{array}{@{}l}
    \CONSTANTS\ \mathit{Node},\ \mathit{Edge}\\
    \VARIABLES\ \mathit{leader}\ \mathit{network}
\end{array}\]
%
for declaring constant sets representing the nodes and edges of a graph and two
variables used in a leader-election algorithm. (When the algorithm is written in
PlusCal, the variable declaration is generated by the PlusCal translator.)

Since \tlaplus is untyped, these declarations do not indicate the types of
constants or variables. (Semantically, all \tlaplus values are sets.) However, a
specification may contain assumptions on the constant parameters such as
%
\[ \ASSUME\ \mathit{Edge} \subseteq \mathit{Node} \times \mathit{Node}  \]
%
and these assumptions are checked by TLC. Properties of variables are expressed
using formulas, typically invariants or temporal logic properties.


\paragraph{Operator definitions.}

The bulk of a \tlaplus module consists in operator definitions. Operators
represent subformulas of the overall specification, including useful operations
on data as well as initial conditions and possible transitions of algorithms.
Again, when using PlusCal, most definitions are generated by the PlusCal
translator. However, a PlusCal algorithm may use operators defined in the
\tlaplus module (or in some extended module), and correctness properties of a
PlusCal algorithm are usually specified in the part of the \tlaplus module below
the generated \tlaplus translation.

An operator definition has the form\footnote{The symbol $\deq$ is written
  \texttt{==} in ASCII.}
\[ \mathit{op}(p_1, \dots, p_n)\ \deq\ e \] 
%
where $p_1, \dots, p_n$ are parameters of the definition and $e$ is a \tlaplus
expression that represents the body of the definition and does not contain
$\mathit{op}$. Each parameter $p_i$ is either an identifier or of the form
$f(\_,\dots,\_)$. The latter case represents an operator parameter, and the
number of underscores indicates the arity of the operator, i.e.\ the number of
arguments that it expects. For example,
\[ \mathit{Apply}(f(\_), x)\ \deq\ f(x)  \]
takes a unary operator parameter $f$ and an ordinary parameter $x$.

\tlaplus enforces the general rule that a name that already exists in the
current context may not be reused. The above definition of $\mathit{Apply}$
would therefore be illegal in a context where $f$ or $x$ have already been
introduced as constant or variable parameters or as operator names. However, the
scope of a parameter symbol is limited to the definition in which it appears,
and it is therefore legal to reuse $x$ as a parameter name in a subsequent
definition.

\paragraph{Recursive operators.}

For the special case of a function definition, one may write
%
\[ f[x \in S] \deq e  \]
%
in order to define a function with domain $S$ that maps every $x \in S$ to the
expression $e$ (which may contain $x$). In this case, $e$ may contain the symbol
$f$. For example, one may write
%
\[
  \mathit{fact}[n \in \mathit{Nat}]\ \deq\
  \begin{array}[t]{@{}l}
    \IF\ n=0\ \THEN\ 1\ \ELSE\ n * \mathit{fact}[n-1]
  \end{array}
\]
%
for defining the factorial function over natural numbers. \tlaplus also allows
for recursive operator definitions; in this case, the names and arities of
recursive operators have to be declared before the definitions, as in
%
\[\begin{array}{@{}l}
    \RECURSIVE\ \mathit{Fact}(\_) \\
    \mathit{Fact}(n)\ \deq\ 
    \IF\ n=0\ \THEN\ 1\ \ELSE\ n * \mathit{Fact}(n-1)
\end{array}\]
%
which defines a recursive \emph{operator} (rather than a function)
$\mathit{Fact}$; note in particular that $\mathit{Fact}$ does not have a domain.
Mutually recursive operators may be introduced by declaring all operator names
and arities before their definitions.

\paragraph{Theorems.}

Finally, \tlaplus modules may assert theorems and contain proofs of these
theorems. Proofs are checked by the \tlaplus Proof System (TLAPS), an
interactive proof assistant for \tlaplus. Theorems are ignored by TLC: instead,
properties to be verified by model checking need to be entered in the TLC pane
of the \tlaplus Toolbox. We will therefore not describe the syntax of theorems
and proofs here.


\subsection{Logic}

\begin{description}
\item[$\land,\ \lor,\ \Rightarrow,\ \equiv,\ \lnot$]\qquad
  \verb|/\|,\ \ \verb|\/|,\ \ \verb|=>|,\ \ \verb|<=>|,\ \ \verb|~|

  are the standard operators of propositional logic (conjunction, disjunction,
  implication, equivalence, and negation).

\item[$=,\ \neq$]\qquad \verb|=|,\ \ \verb|#|

  denote equality and disequality (the latter may also be written \verb|~=|).

\item[$\TRUE,\ \FALSE$]\qquad \verb|TRUE|,\ \ \verb|FALSE|

  are the logical constants true and false.
  
\item[$\forall x: P(x),\ \exists x: P(x)$]\qquad
  \verb|\A x : P(x)|,\ \ \verb|\E x : P(x)|

  represent the logical quantifiers ``forall'' and ``exists''. You may not
  reuse a variable that is already used in the current scope. In other words, if
  $x$ has already been introduced (as a constant, variable, operator or
  parameter of the current operator definition) then it is illegal to write
  $\forall x: P(x)$, but you have to write $\forall y: P(y)$ for some fresh
  variable $y$. There exist ``bounded versions'' of the quantifiers restricted
  to a set $S$ and written as $[\forall|\exists] x \in S: P(x)$, and these are
  the only forms that TLC can evaluate.
  
\item[$\CHOOSE\ x: P(x)$]
  \verb|CHOOSE x : P(x)|

  denotes some arbitrary but fixed value $x$ such that $P(x)$ is true, and some
  unspecified fixed value otherwise. Again, there is a bounded version that
  restricts the choice of possible values to some set $S$, and this is the only
  form accepted by TLC. In mathematical terminology, this operator is known as
  ``Hilbert's $\varepsilon$-operator''. It is important to understand that this
  represents deterministic choice: multiple evaluations of the expression will
  yield the same result. \CHOOSE{}-expressions are most useful if there is a
  unique value satisfying the predicate. For example, the length of a sequence
  $s$ can be defined as
  \[ \CHOOSE\ n \in \mathit{Nat}: \DOMAIN~s = 1\,..\,n \]
  Another useful paradigm is extending some set $S$ by a ``null
  value'':\footnote{The astute reader will notice that this form cannot be
    evaluated by TLC. However, the \tlaplus Toolbox will substitute a special
    ``model value'' for $notAnS$.}
  \[ \mathit{notAnS}\ \deq\ \CHOOSE\ x: x \notin S \]
  The set $S \cup \{notAnS\}$ is then similar to an option type in functional
  programming.
\end{description}

\subsection{Sets}

\begin{description}
\item[$\{e_1, \dots, e_n\}$]\qquad \verb|{e1, ..., en}|

  denotes the set containing (the values corresponding to) the expressions
  $e_1$, \dots, $e_n$. In particular, $\{\}$ is the empty set.

\item[$x \in S,\ x \notin S,\ S \subseteq T$]\qquad
  \verb|x \in S|,\ \ \verb|x \notin S|,\ \ \verb|S \subseteq T|

  is true if $x$ is an element (or is not an element) of set $S$, respectively
  if $S$ is a subset of $T$.
  
\item[$\cup,\ \cap,\ \setminus$]\qquad
  \verb|\cup| (or \verb|\union|),\ \ \verb|\cap| (or \verb|\inter|),\ \ \verb|\|

  set union, intersection, and difference.
  
\item[$\UNION~S$]\qquad \verb|UNION S|

  generalized set union: $S$ is expected to be a set of sets, and the result is
  the union of these sets. For example, $\UNION~\{\{1,2\}, \{3,4\}\} =
  \{1,2,3,4\}$, and more generally, $\UNION~\{A,B\} = A \cup B$.
  
\item[$\SUBSET~S$]\qquad \verb|SUBSET S|

  denotes the set of all subsets of $S$. For example,
  \[ \SUBSET~\{1,2\} = \{ \{\}, \{1\}, \{2\}, \{1,2\} \} \]

\item[$\{x \in S : P(x)\}$]\qquad \verb|{x \in S : P(x)}|

  denotes the subset of elements of $S$ for which the predicate $P(x)$ is true.
  For example, assuming that $\mathit{isPrime}(n)$ is true for a natural number
  $n$ if $n$ is prime then $\{n \in 1\,..\,100 : \mathit{isPrime}(n)\}$ contains
  the set of prime numbers in the interval between $1$ and $100$.
  
\item[$\{e(x) : x \in S\}$]\qquad \verb|{e(x) : x \in S}|

  denotes the set of values obtained by applying the operator $e$ to all
  elements of $S$. This construction is similar to the \verb|map| operator from
  functional programming. For example, $\{2*n+1 : n \in 0\,..\,99\}$ denotes the
  set of the first $100$ odd natural numbers.
  
\item[$\mathit{Cardinality}(S)$]\qquad \verb|Cardinality(S)|

  the number of elements of the finite set $S$. This operator is defined in the
  module \verb|FiniteSets|, which you must \EXTEND{} in order to use it. That
  module also defines the predicate $\mathit{IsFiniteSet}$ for checking if a set
  is finite.
\end{description}


\subsection{Functions}

\tlaplus functions are total over their domain. Although every function is a set
(just like any \tlaplus value), we don't know what set the function denotes and
we think of functions as a primitive class of values. (For nerds: in many
presentations of set theory, functions are sets of pairs. This is not enforced
in \tlaplus.) In terms of programming terminology, functions are analogous to
arrays---although their index set or domain need not be finite---and \tlaplus
uses array syntax for functions.

\begin{description}
\item[\mbox{$[x \in S \mapsto e(x)]$}]\qquad \verb![x \in S |-> e(x)]!

  denotes the function with domain $S$ that maps every element $x$ of $S$ to
  $e(x)$. For example, $[n \in Nat \mapsto n+1]$ is the successor function on
  natural numbers. This expression is similar to a $\lambda$-expression in
  functional programming.
  
\item[$\DOMAIN~f$]\qquad \verb|DOMAIN f|

  denotes the domain of the function $f$. Although \tlaplus has no standard
  notation for the range of a function $f$, it can easily be defined as
  \[ \{f[x] : x \in \DOMAIN~f\} \]
  
\item[\mbox{$f[x]$}]\qquad \verb|f[x]|

  denotes the result of applying the function $f$ to the argument $x$. This
  expression only makes sense if $x \in \DOMAIN~f$.
  
\item[\mbox{$[f\ \EXCEPT\ ![x] = e]$}]\qquad \verb|[f EXCEPT ![x] = e]|

  denotes the function that has the same domain as $f$ and acts similarly as
  $f$, except that it maps the argument $x$ to $e$. For example, if $succ$ is
  the successor function from above, then $[succ\ \EXCEPT\ ![0] = 42]$ is the
  function that maps every natural number to its successor, but that maps $0$ to
  $42$. Within the expression $e$, one may use the symbol $@$ in order to refer
  to $f[x]$. For example,
  \[
    [\mathit{count}\ \EXCEPT\ ![p] = @+1]
  \]
  updates function $\mathit{count}$ by incrementing $\mathit{count}[p]$ by $1$.

  Updating a function at several arguments can be written as
  \[
    [f\ \EXCEPT\ ![x] = a,\ ![y] = b,\ ![z] = c]
  \]
  
\item[\mbox{$[S \rightarrow T]$}]\qquad \verb|[S -> T]|

  denotes the set of all functions whose domain is $S$ and such that $f[x] \in
  T$ holds for all $x \in S$.
\end{description}

For functions that take several arguments, \tlaplus adopts the convention that
$f[x,y]$ is shorthand for $f[\seq{x,y}]$ and hence $f \in [X \times Y
\rightarrow Z]$. This convention extends to \EXCEPT{} expressions, so you can
write $[f\ \EXCEPT\ ![x,y] = e]$.


\subsection{Records}

A \tlaplus record is a function from a finite set of fields (represented as
strings) to values. In principle, standard function operations therefore apply
to records. Nevertheless, \tlaplus introduces some specific syntax for records.

\begin{description}
\item[\mbox{$[\mathit{fld}_1 \mapsto e_1, \dots, \mathit{fld}_n \mapsto e_n]$}]\qquad
  \verb![fld1 |-> e1, ..., fldn |-> en]!

  constructs a record with fields $\mathit{fld}_i$ holding values $e_i$. For
  example,
  \[
    [\mathit{kind} \mapsto \str{request},\
     \mathit{sndr} \mapsto p,\
     \mathit{clk} \mapsto 42]
  \]
  could represent a request message sent by process $p$ with logical clock value 42.

\item[$r.\mathit{fld}$]\qquad \verb|r.fld|

  selects the value of field $\mathit{fld}$ from record $r$.

\item[\mbox{$[r\ \EXCEPT\ !.\mathit{fld} = e]$}]\qquad
  \verb|[r EXCEPT !.fld = e]|

  denotes the record that is similar to $r$, except that field $\mathit{fld}$
  holds value $e$. (The indicated field should exist in record $r$.) An
  analogous generalization to the \EXCEPT{} construct for functions allows
  several record fields to be updated.
\end{description}


\subsection{Numbers}

You must import the arithmetical operators by \EXTEND{}ing one of the standard
modules \verb|Naturals| or \verb|Integers|. (\tlaplus also has a standard module
\verb|Reals| for real numbers, but we won't need it.)

\begin{description}
\item[$\mathit{Nat},\ \mathit{Int}$]\qquad \verb|Nat|,\ \ \verb|Int|

  denote the set of natural numbers ($0, 1, \dots$) and of integers.
  
\item[$i\,..\,j$]\qquad \verb|i .. j|

  an interval of integers, such as \verb|-3 .. 5|.
\item[$+,\ -,\ *,\ \div,\ \mathit{mod}$]\qquad
  \verb|+|,\ \ \verb|-|,\ \ \verb|*|,\ \ \verb|\div|,\ \ \verb|%|

  are the standard arithmetical operations (addition, subtraction,
  multiplication, integer division and modulus).
\end{description}


\subsection{Tuples and sequences}

Tuples and sequences are the same mathematical objects, but they are used
differently: whereas a tuple has a fixed number of components, a sequence can be
of varying length. In \tlaplus, tuples and sequences are represented as
functions with domain $1\,..\,n$, for some natural number $n$. (In particular,
the empty sequence has domain $1\,..\,0$.) Standard function operations
therefore apply, and the $i$-th element of a sequence $s$ is obtained as $s[i]$.
The sequence operations must be imported by \EXTEND{}ing the standard module
\verb|Sequences|.

\begin{description}
\item[$\seq{e_1,\dots,e_n}$]\qquad \verb|<<e1, ..., en>>|

  denotes the sequence with elements $e_1$, \dots, $e_n$, in that order. In
  particular, \verb|<< >>| is the empty sequence.
  
\item[$s \circ t$]\qquad \verb|s \o t|

  denotes the concatenation of the sequences $s$ and $t$. For example,
  $\seq{1,2} \circ \seq{3,4} = \seq{1,2,3,4}$.
  
\item[$\mathit{Seq}(S)$]\qquad \verb|Seq(S)|

  the set of all finite sequences with elements in set $S$.
  
\item[$S \times T$]\qquad \verb|S \X T|

  denotes the set product of $S$ and $T$, i.e.\ the set of all pairs $\seq{s,t}$
  with $s \in S$ and $t \in T$.
  
\item[$\mathit{Len}(s)$]\qquad \verb|Len(s)|

  denotes the length of sequence $s$.
  
\item[$\mathit{Append}(s,e)$]\qquad \verb|Append(s,e)|

  adds element $e$ at the end of the sequence $s$.

\item[$\mathit{Head}(s),\ \mathit{Tail}(s)$]\qquad
  \verb|Head(s)|,\ \ \verb|Tail(s)|

  denote the first element of the sequence $s$, respectively the rest of the
  sequence with the first element removed. These operators are well-defined only
  if the sequence $s$ is non-empty.
  
\item[$\mathit{SubSeq}(s,m,n)$]\qquad \verb|SubSeq(s,m,n)|

  the sequence $\seq{s[m], s[m+1], \dots, s[n]}$.

\item[$\mathit{SelectSeq(s, P(\_))}$]\qquad \verb|SelectSeq(s, P(_))|

  denotes the subsequence of elements of sequence $s$ that satisfy predicate
  $P$. For example, if $\mathit{isPrime}(n)$ is true if $n$ is a prime number,
  then
  \[ \mathit{SelectSeq}(\seq{2,3,4,5,6,7,8}, \mathit{isPrime}) =
     \seq{2,3,5,7} \]
\end{description}

Many more operators on sequences can easily be defined in \tlaplus. For example,
the ``map'' operation on sequences is defined as
\[
  \mathit{Map}(s, f(\_))\ \deq\
  [i \in \DOMAIN~s \mapsto f(s[i])]
\]
Other operations require recursive definitions, such as a ``reduce'' operation
that applies an operation $op$ to all elements of the sequence $s$ starting from
the ``null value'' $e$:
\[\begin{array}{@{}l}
    \RECURSIVE\ \mathit{Reduce}(\_,\_,\_)\\
    \mathit{Reduce}(s, \mathit{op}(\_,\_), e)\ \deq\
    \begin{array}[t]{@{}l}
      \IF\ s = \seq{}\ \THEN\ e\\
      \ELSE\ \mathit{op}(\mathit{Head}(s), \mathit{Reduce}(\mathit{Tail}(s), \mathit{op}, e))
    \end{array}
\end{array}\]

Character strings are represented as sequences of characters. How characters are
represented is left unspecified; literal strings are written as \verb|"string"|.
However, TLC handles strings natively and does not support sequence operations
applied to strings.


\subsection{Miscellaneous constructs}

\begin{description}
\item[$\IF\ P\ \THEN\ t\ \ELSE\ e$]\qquad \verb|IF P THEN t ELSE e|

  is a conditional expression. It can be used as a formula if $t$ and $e$ are
  both formulas, but also more generally as a term. See the definition of the
  factorial function earlier for an example.
  
\item[$\CASE\ p_1\ \rightarrow\ e_1\ \Box\ p_2\ \rightarrow\ e_2\ \dots \Box\ \OTHER\ \rightarrow\ e$]\mbox{}\\
  \verb|CASE p1 -> e1 [] p2 -> e2 ... [] OTHER -> e|

  generalizes conditional expressions to more than two branches. It equals some
  $e_i$ such that $p_i$ is true, or $e$ if no guard $p_i$ is true (The \OTHER{}
  branch is optional.) In case several $p_i$ are true, the choice of which
  branch is evaluated is fixed but unspecified: make sure that in this case, all
  corresponding $e_i$ evaluate to the same value.
  
\item[$\LET\ x \deq t\ \IN\ e$]\qquad \verb|LET x == t IN e|

  allows users to introduce local definitions, similar to the standard ``let''
  construct of functional programming languages.
\end{description}


\section{Online Resources}

\begin{itemize}
\item \url{learntla.com}: A Web site for helping newcomers learn PlusCal and
  \tlaplus. There is also a published book (``Practical \tlaplus'') by the same
  author with a more complete introduction.
\item \url{http://lamport.azurewebsites.net/tla/tla.html}: The official \tlaplus
  Web site with lots of reference material.
\item \url{http://lamport.azurewebsites.net/video/videos.html}: A video course
  on \tlaplus by Leslie Lamport (does not cover PlusCal, though).
\item \url{http://lamport.azurewebsites.net/tla/c-manual.pdf}: The PlusCal
  manual for the C syntax that is also used in the lectures.
\end{itemize}

\end{document}
